\title{Formulae for Oort Lore}
\author{Odradek}
\date{\today}

\documentclass[10pt]{article}

\pagestyle{headings}

\usepackage[margin=3cm]{geometry}
\usepackage[colorlinks=true, linkcolor=blue]{hyperref}

\begin{document}
	\maketitle
	\newpage
	
	\tableofcontents
	\newpage
	
	\begin{abstract}
		Some derivation for formulas that can be useful to calculate stuff in space.
	\end{abstract}
	
	\section{Notation}
	
	\begin{itemize}
		\item $s$: Distance
		\item $v_0$: Velocity at the beginning
		\item $v_{end}$: Velocity at the end
		\item $a$: Acceleration
		\item $a_{eff}$: Effective acceleration (any gravitational has been removed: $a_{eff} = a - g$)
	\end{itemize}
	
	When we need them in vector form, they are written as $\vec{s}$ for example.
	
	\section{Travel Time Without Relativity}\label{TravelTime}
	
	\subsection{Constant speed}
	
	From 
	
	\begin{equation}
		s = vt
	\end{equation}
	
	follows

	\begin{equation}
		t = s/v.
	\end{equation}
	
	\subsection{Under acceleration}
	
	From $s = 0.5at^2$ follows $t = \sqrt{2s/a}$ and a final velocity of $v=at$.
	
	\subsubsection{With start velocity}
	
	From $s = v_0t + 0.5at^2$ follows $t = \frac{-v_0 + \sqrt{v^2_0 + 2sa}}{a} \rightarrow s/v_0$ for $a\rightarrow0$ (l'Hospital) and a final velocity of $v_1 = v_0 + at$.
	
	Note that the last formula can also be written as $v_1=v_0 + (-v_0 + \sqrt{v^2_0+2sa}) = \sqrt{v^2_0+2sa}$, from which we can get the final velocity without calculating the time first.
	
	\subsection{Example}
	
	In our example we want to calculate the time and end velocity for a travel from earth to mars with two different outcomes: First we want to go to mars with a constant burn of $1g$ after we have escaped earth. Second we want to arrive at mars with a velocity of zero, but in the shortest time possible with a maximum acceleration of $1g$ after having escaped earth. In both scenarios, mars is supposed to be only $s_m = 0.4 AU = 60 000 000 km$ away from earth, which is still the more than the shortest possible distance.
	
	\paragraph{Phase one.}
	Let's assume we want to escape earth with an effective acceleration of $a_{eff1} = a-g = 4g = 0.04 km/s^2$ until we reach a height of $s_1=5000 km$, which is half of the exosphere. We just assume the computer keeps the effective acceleration at a constant $4g$.
	
	Then we reach the first destination after $t_1 = \sqrt{2s_1/a_{eff1}} = sqrt(10000km/(0.04km/s^2)) = 500 s = 8.3 min$ with a velocity of $v_1=a_{eff1}t_1=0.04km/s^2\cdot 500s = 20 km/s$, which will be our start speed for the next phase.
	
	\paragraph{Phase two.}
	Now we want to accelerate with a burn of $a_{eff2}=2g=0.02km/s$ until we leave the exosphere. For this, we need to pass another $s_2=5000km$. Mind that we now have a non-zero start velocity of $v_1 = 20 km/s$.
	
	Then we reach the second destination after $t_2 = \frac{-v_1+\sqrt{v^2_1+2s_2a_{eff2}}}{a_{eff2}} =\frac{-20+\sqrt{400+200}}{0.02} \approx 224.7 s \approx 3.7 min $ with a final velocity of $v_2 = v_1 + a_{eff_2}t_2 = 20 km/s + 0.02 km/s^2\cdot 224.7 s = 24.5 km/s$.
	
	\paragraph{Phase three, first scenario.}	
	Now we want to go to mars and have $s_3 = 60 000 000 km, v_2 = 24.5 km/s$ and $a_{eff3} = 0.01 km/s^2$.
	
	Let's calculate the final velocity first, this time. It is $v_3 = \sqrt{v^2_2 + 2sa_{eff3}} = \sqrt{600 + 1 200 000} = 1096 km/s$, which is about $0.3 \%$ of the speed of light and we expect not much relativistic effects here. With this, we can also calculate the time and we get $t_3 = \frac{-v_2 + v_3}{a_{eff3}} = \frac{-24,5 + 1096}{0,01} = 107150 s = 29.8 h$. So less then thirty hours to get to mars this way. Nice!
	
	\paragraph{Phase three, second scenario.}	
	This time, we need to make sure that our final velocity $v_{end}$ is zero. We assume that the optimal journey is is achieved by an acceleration of $a_{eff4} = 1g$ for the first part, and a deceleration of $a_{eff5} = -1g$ for the second part. Due to our non-zero start velocity, the change won't be exactly at fifty percent.
	
	Let's first look at the distance: $s_3 = v_2 t_4 + v_3t_5 + 0.5(a_{eff4}t^2_4 + a_{eff5}t^2_5)$ with the two unknowns $v_3, t_4$ and $t_5$. We can split this up into $s_4 = v_2 t_4 + 0.5a_{eff4}t^2_4$ and $s_5 = v_3 t_5 + 0.5a_{eff5}t^2_5$, now with the five unknowns  $s_4, s_5, v_3, t_4$ and $t_5$.
	
	Our equation for the final velocity also turns into two equations: $v_3 = \sqrt{v^2_2+2s_4 a_{eff4}}$ and $v_4 = \sqrt{v^2_3+2s_5 a_{eff5}}$.
	
	Now let's start rolling this up from the back: $v_4 = \sqrt{v^2_3+2s_5 a_{eff5}} = \sqrt{v^2_2+2s_4 a_{eff4}+2s_5a_{eff5}} = 0 \rightarrow s_5 = \frac{v^2_4-v^2_2-2s_4a_{eff4}}{2a_{eff5}} = \frac{v^2_2+2s_4}{2}$. Now let's insert $s_4 = s_3 - s_5$ and we get $s_5 = \frac{v^2_2 + 2(s_3 - s_5)}{2} \rightarrow 2s_5 = \frac{v^2_2 + 2s_3}{2} = 24.5^2/2 + 60 000 000 = 60 000 300 km \rightarrow s_5 = 30 000 150 km, s_4 = 29 999 850 km$.
	
	From this, we can calculate the times and final velocities. $v_3 = \sqrt{v^2_2 + 2s_4 a_{eff4}} = \sqrt{600 + 599 997} = 775 km/s, v_4 = \sqrt{v^2_3 + 2s_5 a_{eff5}} = \sqrt{600 625 - 600 003} \approx 24.5$. There must be a mistake somewhere.
	\section{Travel Time With Special Relativity}
\end{document}